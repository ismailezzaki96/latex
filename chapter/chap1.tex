L’astrophysique est une science dont l’objet est l’étude physique des corps célestes. Elle
se propose d’interpréter et d’unifier les données observées par la recherche astronomique
en élaborant des lois physiques qui peuvent les expliquer \cite{2}. On s’intéresse dans ce
chapitre d’étudier les trous noirs en Astrophysique.
	\section{Formation des trous noirs}
	
	Les étoiles, à la fin de leur vie, connaissent des destins très différents dont la nature
dépend de la masse initiale de l’étoile. En effet, les réactions de fusion nucléaire qui ont lieu
dans les noyaux des étoiles produisent des éléments de plus en plus lourds en commençant
par l’hydrogène.
Une étoile peu massive comme le soleil, ne peut pas aller très loin dans la fusion,
son noyau se contracte pour devenir une naine blanche de masse inférieure à environ 1,4
masses solaires (limite de Chandrasekhar).


Quand aux étoiles plus massives, qui explosent sous forme de supernova du type II 1 .
Les parties extérieures de l’étoile se dispersent dans l’espace, alors que son noyau s’effondre
complètement sous son propre poids. Si après la supernova de type II, le noyau restant
est très massif $M bigger than M_Soleil$ , aucune force répulsive connue à l’intérieur d’une étoile ne
peut repousser assez fort pour empêcher l’étoile à s’effondrer sur lui-même en quelques
secondes. La matière se comprime rapidement, et cette matière dense forme le trou noir,

Quand le trou noir est formé, il faudrait une vitesse supérieure à la vitesse de la lumière
pour s’échapper de son champ gravitationnel, qui est très intense. Aucun objet ne peut
atteindre une vitesse plus grande que celle de la lumière, alors aucune matière ou radiation
ne peut s’échapper.

N’importe quel objet ayant une masse M suffisamment comprimée, peut devenir un
trou noir. Il suffit d’avoir un rayon égale au rayon de Schwarzschild correspondant.

	
	
	
	
	
	
	
	
	
	Pour la plupart des personnes qui savent ce qu’est un trou noir, il pense que l’origine d’un trou noir est absolument la mort d’une étoile or tous les astres de l’univers peuvent former un trou noir, en effet, pour former un trou noir il faut une masse élevée .Prenons tout d’abord une étoile de taille moyenne c'est-à-dire dont la masse est à peu près égale à la masse du Soleil (de 0.5 à 4 masses du Soleil). Une fois que l’étoile à utiliser tout son réservoir en hydrogène l’étoile entame son réservoir en hélium mais le rythme de la fusion nucléaire de l’hélium pour donner du carbone s’accélère pour donner une étoile appelée Géante rouge.  Ensuite la vie de l’étoile n’est pas encore complètement finie en effet après avoir utiliser toutes ces réserves en hélium pour former du carbone et de l’oxygène. A ce stade les étoiles sont alors dépourvu de toutes les couches de gaz autour d’elle alors le noyau de celle-ci commence alors à se refroidir pour obtenir une faible température , cette étoile devient alors une Naines Blanches.\\ 
	Si on prend une étoile de masse comprise entre quatre et dix masses du Soleil environ alors on obtient une Super Géante rouge qui ont les mêmes propriétés que les Géantes Rouges mais avec une masse supérieure à dix masses solaires. Quand les Super Géantes commencent à ne plus avoir assez de combustibles pour faire des réactions de fusion thermodynamique, celles-ci commencent alors à s’effondrer sur elles mêmes et à cause de leur très grande masse ces étoiles parviennent alors à former de très grandes réactions qui par leur force forment alors des supernovas, les restes de la Super Géantes alors se trouvent dans l’univers comme ceux-ci sont très proches et de très faibles tailles, puisque ce sont des protons et des électrons, il se recombinent pour former des neutrons et ainsi continuer à « vivre » pour former une étoile à neutrons de très faibles diamètres (environ 20 km) elles sont aussi appelées pulsars.\\
	Maintenant prenons une étoile dont la masse est supérieure à dix masses solaires alors quand l’étoile commence à ne plus avoir assez de carburant (hélium et hydrogène) celle-ci s’effondre mais ne laisse pas place à une supernova et une étoile à neutron comme les Super Géantes en effet celle-ci s’effondre mais comme leur gravité est très importante puisque leur masse est aussi très grande alors elle s’effondre et comme elle n’a plus de carburant elle ne peut plus effectuer une réaction chimiques pour s’enlever de sa force de gravité et de plus sa taille diminue aussi à grande vitesse jusqu’à atteindre la limite de d’Oppenheimer- Volkoff.\\
	À partir de cette limite l’effondrement est tel que plus aucune particule ou autre rayon comme la lumière ne peut s’en échapper et on assiste alors à la formation d’un trou noir.\\
	En astrophysique, la limite d'Oppenheimer-Volkoff, du nom de deux physiciens qui la calculèrent la première fois, représente la masse maximale théorique que peut avoir une étoile à neutrons. Au-delà de cette valeur, l'objet s'effondre alors en trou noir. La valeur de cette limite est d'environ 3,3 masses solaires et est à comparer avec la limite de Chandrasekhar pour les naines blanches. Cette limite est la valeur de la masse maximum du coeur de l'étoile \cite{3}.
	\begin{center}
		
	\end{center}
	Figure 1.1 –Les étapes de formation d’un trou noir
	\section{Les différents types de trous noirs}
	En fonction de leur masse, on distingue quatre types de trous noirs.
	\subsection{ trous noirs stellaires }
	La grande majorité des trous noirs seraient d’origine stellaire, c'est-à-dire de l’effondrement gravitationnel d’une vieille étoile massive sur elle-même. Ceux-ci ont une masse d'au moins quelques masses solaires. Ce type de trou noir ne fait que quelques kilomètres de diamètre. Leur formation peut engendrer des ondes gravitationnelles.
	Les principaux progéniteurs de trous noirs stellaires par effondrement sont les étoiles Wolf-Rayet qui est une étoile chaude, massive et évoluée présentant un taux de perte de masse très élevé . 
	\subsection{ Les trous noirs supermassifs }
	La formation des trous noirs supermassifs est encore fortement débattue car elle se fait sur de grandes échelles de temps (contrairement à la formation d’un trou noir stellaire). Comme il n’existe pas d’étoile de masse si grande, les trous noirs supermassifs ne peuvent pas directement être conçut d’un effondrement stellaire.Il pourrait s’agir d’une étoile massive qui s’effondre et qui donne naissance à un trou qui grandit peu à peu en se nourrissant d’autres étoiles ou bien d’un énorme nuage de gaz qui s’écroule directement sous sa propre gravité. Bien que l’origine des trous noirs supermassifs ne soit pas clairement définie, leur existence est en tout cas tout à fait possible.\\ 
	En astrophysique, un trou noir supermassif est un trou noir dont la masse est d’environ un million à un milliard de masse solaires .La densité de ce genre de trou noir  est très faible (parfois plus faible que celle de l’eau), des études montrent que plus un trou noir est grand, plus sa densité diminue, même si sa masse croit sans limite.
	\begin{figure}[H]
			\begin{center}
		
	\end{center}

	 \caption{Le trou noir supermassif est situé au centre de la galaxie .}
	
	\end{figure}
	\subsection{ Les trous noirs intermédiares }
	Les trous noirs intermédiaires sont des objets récemment découverts, de masse entre 100 et
	10 000 masses solaires. On peut donc dire que les trous noirs intermédiaires ne peuvent pas se former par simple effondrement d’étoiles massives.\\
	En Novembre 2004,une équipe d’astronome découvre le premier trou noir intermédiaire, orbitant à 3 années-lumières seulement au centre de notre galaxie. C’est un trou noir de 1 300 masses solaires. Grâce à ces observations, nous pouvons dire que les trous noirs de masse intermédiaire jouent un rôle dans la formation des trous noirs supermassifs.
	\subsection{ Les trous noirs primordiaux } 
	Ce type de trou noir ne peut pas être expliqué de la même manière que les trois précédents puisque celui-ci n'a jamais été vraiment prouvé, ce n'est que une hypothèse.\\ 
	Leur existence, toujours hypothétique, remonterait au début de la création de notre univers, il y a de cela 13,7 milliards d'années, bien avant l’émission de la première lumière de l'Univers.Son Masses Solaires n'ont pas été totalement définies, elles sont juste bien plus faible que celle des autres trous noirs.\\
Contrairement aux autres trous noirs, les trous noirs primordiaux perdent leur masse
de plus en plus rapidement et finissent par disparaître. Ce phénomène a été baptisé "évaporation quantique " par Stephen Hawking en 1975 \cite{4}.
	\section{Détection des trous noirs } 
	Les trous noirs, par leur caractère « invisible » ne se détectent que par leurs effets sur l’environnement. On peut distinguer deux grandes catégories de méthodes de détection :
	- quand le trou noir est accompagné d’une étoile « normale », on parle alors	
	de système binaire.
	- quand le trou noir est seul, on le dit alors célibataire.	
	\subsection{ Trou noir dans un système binaire }
	Etant donné son fort champ gravitationnel, le trou noir peut avoir une étoile comme satellite. C’est l’étude de la lumière émise par cette dernière qui nous permet de détecter le trou noir : dans un système binaire, les deux astres tournent l’un autour de l’autre. Lorsqu’on mesure le spectre infrarouge de l’étoile, on s’aperçoit qu’il varie périodiquement. Ceci est une application de l’effet redshift.
	Ce spectre prouve que l’étoile tourne autour d’un objet massif et invisible qui peut être soit une naine blanche, soit une étoile à neutrons, soit un trou noir. Pour faire la distinction, on mesure la masse de l’astre invisible en analysant son spectre, comme les étoiles à neutrons et les naines blanches ont une masse limite, si cette dernière est dépassée, le compagnon invisible est un trou noir.\\
	\begin{figure}[H]
		\begin{center}
			\centering

\caption{Conséquences de la variation de la longueur d'onde sur le spectre d'une étoile .}

	\end{center}
\end{figure}

	
	Il existe un autre moyen de détecter les trous noirs dans des systèmes binaires. En effet, lorsqu’une étoile est proche d’un trou noir, elle lui cède de sa matière. Cette matière est inexorablement attirée par le trou noir et tourne autour de celui-ci en formant un disque d'accrétion. En se rapprochant de la singularité, la matière s’échauffe et émet des rayons X. Cette émission est aléatoire car le disque d’accrétion, par son extrême chaleur, est très instable ; il se produit alors des « bulles chaudes » provoquant des sursauts de rayons X \cite{5}.
	Pour différencier l’étoile à neutrons du trou noir, on doit observer le centre du disque d’accrétion :\\
	- celui d’une étoile à neutrons est lumineux en raison de la matière	
	qui heurte la surface de l’étoile effondrée.\\
	- par contre, celui d’un trou noir sera sombre car la matière aura été	
	aspirée et donc aucune lumière ne nous arrivera.
	\begin{figure}[H]
		
	
	\begin{center}
		
	\end{center}
	\caption{Différence entre un trou noir et un étoile à neutrons.} 
\end{figure}
	
	\subsection{ Trou noir célibataire }
	Un trou noir célibataire est difficile à détecter ; le meilleur moyen est d’utiliser ses propriétés liées à la lumière, et notamment l'effet de lentille gravitationnelle.
	
	Les trous noirs dévient la trajectoire des rayons lumineux , c'est pourquoi on peut avoir deux images identiques d'une même étoile située derrière le trou noir.\\
	\begin{figure}[H]
	\begin{center}
		\centering
			
	\end{center}
\caption{l'effet lentille gravitationnelle.}
\end{figure}
	
	Dans la réalité, les deux images de l’étoile sont très proches voir confondues ce qui donne une étoile très lumineuse qui, si elle est détectée, nous informera sur la présence d’un corps céleste qui peut s’avérer être un trou noir.
	Comme pour les trous noirs dans les systèmes binaires, on peut les détecter grâce à leur rayonnement X. En effet, le trou noir célibataire possède lui aussi un disque d’accrétion : la méthode utilisée pour la détection d’un trou noir en système binaire peut donc être appliquée. Cependant, le disque d’accrétion d’un trou noir célibataire est très faible et devient indétectable par nos instruments de mesures s'ils sont distants de plus de 10 années-lumière.
	La détection d’un trou noir célibataire reste donc très théorique, même si on a réussi à observer un exemple flagrant de lentille convergente gravitationnelle.
	
	\section{Propriétés des trous noirs }
	Quasi toutes les propriétés des objets tombés dans le trou noir disparaissent. Seules subsistent trois propriétés : la masse, la charge électrique et le moment angulaire. C’est le théorème de la calvitie démontré par Werner Israel en 1967 : un trou noir (on dit qu'il n'a pas de poils) est entièrement connu par ces trois caractéristiques.\\
	On a dès lors quatre trous noirs possibles : celui qui a une masse sans charge électrique et qui ne tourne pas sur lui-même (trou noir de Schwarszchild), celui qui a une masse plus une charge électrique et qui ne tourne pas sur lui-même  (trou noir de Reissner-Nordström) et celui qui a une masse sans charge et un moment cinétique ( trou noir de Kerr),et enfin celui qui a une masse ,une charge et un moment cinétique (trou Kerr Neumann) quatre solutions exactes des équations de la relativité (voir le chapitre 2). 
		
		\begin{table}[H]
			\begin{center}
			\centering
		
			{ \renewcommand{\arraystretch}{1.9}
				
				\begin{tabular}{|l|l|l|}
					\hline
					& $J=0$ & $J\neq 0$\\
					\hline
					$Q=0$ & Schwarzschild &  Kerr \\
					
					\hline
					$Q\neq 0$ & Reissner-Nordstrom  & Kerr-Neweman \\
					\hline
					
					\hline
					
			\end{tabular}}
			\end{center}
			\caption{Les types théoriques du trou noir.}
		\end{table}
		
	
	\newpage 	
