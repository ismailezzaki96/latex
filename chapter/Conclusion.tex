 
Les trous noirs sont des objets célestes sur lesquels il reste encore beaucoup de questions en suspend. Dans ce modeste mémoire, nous avons essayé d'approcher du notion du trou noir ,il est défini en astrophysique  comme un objet massif dont le champ gravitationnel est si
s'intense qu'il empêche toute forme de matière ou de rayonnement de s'en échappé .On a présentée aussi la formation de ces objets, puis leurs classifications, et on a fini par quelques
méthodes de détection des trous noirs, même s'ils sont invisibles, ils ont une énorme influence sur la matière qu'ils entourent.\\
 Ainsi, nous avons traité les trous noirs dans la
théorie la plus compatible à décrire ce genre des objets, c'est la relativité générale, en se
basant sur la notion de l'espace-temps,qui basé sur  l’espace-temps pour décrire le fort champ gravitationnel produit
par les trous noirs, en donnant la métrique des différents types théoriques du trou noir qui sont
définis par trois paramétres : la masse, la charge et le moment cinétique. On a défini ainsi les
différentes régions caractéristiques du trou noir \\
Nous nous sommes intéressés dans la suite à quelques rudiments de la thermodynamique des
trous noirs. On a ainsi pu voir que l’affirmation "rien ne peut sortir d’un trou noir" est en réalité
fausse, puisque Hawking a pu mettre en évidence la présence d’un rayonnement de particules
qui ressemble à celui d'un rayonnement thermique. Nous avons aussi cités les quatre lois qui gouverne la dynamique des trous noirs, l’expression
de la température de Hawkinge et l’entropie de Bekeinstein-Hawking.\\
Ensuite, nous avons étudié le comportement critique d'un trou noir de Schwarzschild AdS,anssi nous avons exprimés leurs expressions des
différentes grandeurs thermodynamiques ,la stabilité thermodynamique a été bien étudié en utilisant la capacité thermique.\\
De m\^{e}me on a étudié la thermodynamique d’un trou noir AdS chargé dans un espace de phase
étendu, en traitant la constante cosmologique et sa quantité conjuguée, comme des variables thermodynamiques associées à la pression et au volume, respectivement. Pour une
charge de trou noir Q fixe, cette identification nous a permis d’écrire l’équation d’état
comme suit P = P(V, T) et d’étudier son comportement en utilisant les techniques thermodynamiques standard. La stabilité thermodynamique a été bien étudié en utilisant la
capacité thermique.
Finalement, on a étudié la géothermodynamique d’un trou noir RN-AdS en se basant
sur la géométrie de Ruppeiner Quevedo et HPEM et on a montré que seul Quevedo et
Ruppeiner sont capable de décrire la transition de phase thermodynamique
