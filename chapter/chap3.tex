 

Black hole thermodynamics has been an intriguing subject of discussion for decades. The analogy between space–time with black hole horizon and thermodynamics have been extensively investigated (Bardeen et al. 1973). The four laws of black hole thermodynamics were discovered by Carter, Hawking, and Bardeen (Bardeen et al. 1973). 
 

Classiquement rien ne peut s'extraire d'un trou noir, cette idee a amene Wheere en 1970 à remarqué que l'existence d'un trou noir dans le cadre de la théoric classique de la gravitation n'est pas compatible avec les lois themodynamiques habituelles, en effet Wheere a imagine l'expcrience de pensce suivante: a un instant initial $\mathrm{t}=0$ un trou noir -suppose sans entropie-de masse $M$ est $\mathrm{cn}$ -
du systeme forme par le trou noir et la matière environnante est $E_{t o t}=E+M c^{2}$ et 1 'entropie total du système est $S_{\text {tot }}=S+0,$ a un instant ultérieur la matiere est absorbée par le trou noir et on aura $E_{t o t}=E+M c^{2}$ et $S_{t o t}=0 .$ Cette experience de pensee montre que l'entropie $S_{t o t}$ du système diminue au cours du temps ce qui viole le second principe de la thermodynamique. Comme le comprit Bekenstein, il devient necessaire d'associer une entropie a un trou noir pour resoudre ce paradoxe, et cette disparition de l'entropie peut etre evite si on considere l'entropie genéralisce:
$$
S=S_{T N}+S_{e x t}
$$
$S_{T N}$ est $1^{\circ}$ entropie du trou noir et $S_{e x t}$ l'entropie du milieu extericur du trou noir. Bekenstein suggére que l'entropie gencralise ne peut que croitrel22].

Après la proposition de Wheere, une scric de theoremes ont prouve que les horizons d' çvenements des trous noirs presentent une analogie surprenante avec les lois de la thermodynamique ordinaire. Nous avons dejà rencontre un tel signal dans le theoreme d'Hawking sur l'aire de I'horizon d'un trou noir, ce théorime suggere une analogie formelle avec le deuxieme principe de la thermodynamique.



\section { Rayonnement du trou noir}
\subsection {Effet Hawking}
Nous venons de présenter une analogie entre les lois de la thermodynamique des trous noirs et les principes de la thermodynamique ordinaire. Par exemple, si nous remplasons formellement dans le premier principe de la thenmodynamique $E$ par $M, T$ par $\frac{\kappa}{2 \pi}$ et $S$ par $d$, alors nous retrouvons le premier principe de la thermodyminique des trous noirs  De méme on a vue que l'air de l'horizon d'un trou noir ne peut qu'augmenter au cours du temps, par analogie avec le second principe.
 Pourtant certains auteurs soulignent que cette analogie est purement mathématique, les trous noirs ne sont pas des objets thermodythamiques, parce qu'un trou noir est vide de matiere, il ne fait qu'absorber et n'émet rien $(T=0)$

Jacob Bekenstein affirme que lanalogie avec la thermodynamique est bien de nature physique, Cotte interprétation fut énergiquement combattue par S. Hawking jusqu'au jour, en $1974,$ où il prit en compte les effets de la mécanique quantique jusqu'alors ignorés. A sa grande surprise, ses calculs indiquèrent que I l'hypothèse thermodynamique de J. Bekenstein était parfaitement fondée [36] :
les trous noirs émettent bien une radiation".
En effet, tout d'abord il faut comprendre que le vide $\mathrm{cn}$ fait n'est pas vide. Des particules sont erées en permanence par paire (particules et anti-particules).

Par analogie avee l'effet tunnel, on peut sattendre à ce qu'une trajectoire classique-
cependant autorisée au niveau quantique. Suivant cette idee, il a été montré par $\mathrm{Hartle } ~$ et Hauking que la probabilité quantique de sortir de l'intérieur d'un tron noir n'est pas nulle. De facon plus quantitative, il est possible d'établir que la distribution en énergie (i.e, la densité d'états quantiques) de particules traversant l'horizon par effet tumel et s'échappant du trou noir sera une distribution presque thermique de corps noir.

Ainsi, le principe d'incertitude oblige, un tel état physique est perpétuellement animé de fluctuations quantiques sous la forme de création puis annihilation de paires particule/antiparticule. Ces paires s'annihilent généralement très rapidement. Au voisinage de I'horizon d'un petit trou noir, là où la courbure est gigantesque, leffet de marce (en termes classiques) peut cependant permettre de séparer les deux particules de la paire. L'une entre vers le trou noir et lautre est éjectée vers l'infini [11] , figure 3.1 .
3.4.2 Evaporation
Dans l'effet Hawking l'une des particules créces ( 2 savoir, la particule qui a l'energie négative) est cree sous l'horizon de l'evénement. tandis que l'autre, avec l'energie positive est crée en dehors de l'horizon. Le rayonnement Hawking emporte l'energie, et par conséquent la masse du trou noir diminue. Cette observation, basce sur la conservation de l'énergie, implique qu il doit y avoir un flux d'energie négative à travers l'horizon dans le trou noir. Cela peut arriver seulement si la movenne quantique de tenseur d'énergicimpulsion $T_{\mu \nu}$ viole la condition d'energie faible, nous souvent suppose que la condition d'énergie faible est satisfaite. Il y avait des raisons de croire en cela alors que nous avions affaire à des systemes classiques et des processus classiques. Maintenant, quand nous commencons la consideration des aspects quantiques de la physique du trou noir, nous pourrions nous attendre à ce que certains des résultats prouvés plus tôt ne soient pas directement applicables. La consequence la plus importante est la violation de la théorème des airs de Hawking dans le domaine quantique (Markov 1974) $\mid 37]$. Dans le processus de creation de particules quantiques la masse (et done la surface) d'un trou noir diminue. Ce processus est connu sous le nom " tévaporation d'un trou noir "[38], figure 3.2.
\subsection {Luminosité d'un trou noir}
Comme on a dejá signale, le savent Steven Hawking donne des expressions pour l'entropie
$$
S=\frac{k_{b} c^{3} A}{4 h G}
$$

ot la temperature
$$
T=\frac{h c^{3}}{8 \pi k_{b} G M}
$$
Avec $\mathcal{A}$ est l'air de l'horizon des evénements, $k_{b}$ est la constante de Boltzmann, $h=\frac{h}{2 \pi}$ ( $h$ est. la constante de Plank Connaissant le rayon de Schwarzschild, on peut calculer l'air de l'horizon $\mathcal{A}$ par
$$
A=4 \pi a^{2}=\frac{16 \pi G^{2} M^{2}}{e^{4}}
$$
Comme conséquence de l'évaporation, la diminution de masse du trou noir. On peut parler done de la luminosite et la duree de vie de ce dernier.

Gràce aux résultats précédentes, la luminosito de tayommement d'Hawking est donnée par
$$
L=A \sigma T^{4}=\frac{\hbar c^{2}}{3840 \pi a^{2}}=\frac{\hbar c^{6}}{15639 \pi G^{2} M^{2}}
$$
\subsection { Durée de vie d'un trou noir}
L'angmentation de la tempémture avec la perte de masse indiquée dans l'équation
(3.15) staggere qu'avec le temps, la vitesse a laquelle l'energie est emise par le trou noir devrait également augmenter. Après plus de réarrangement, un taux de perte de masse donne
$$
\frac{d m}{d t}=\frac{h c^{4}}{15360 \pi G^{2} M^{2}}
$$
Pour trouver la durée de vie de trou noir $\tau$ en fonction de la masse initiale $M$ du trou noir, on sépare les variables de l'équation (3.18) et on intègre entre un temps initial supposant égale 0 et un certain temps $\tau$. on obtient
$$
\tau=\int_{0}^{\tau} \mathrm{d} t=-\frac{15360 \pi G^{2}}{h c^{4}} \int_{0}^{M} M^{2} \mathrm{d} M
$$
Finalement, la durée de vie de tron noir de masse initiale $M$ est égale
$$
\tau=\frac{5120 \pi G^{2}}{h c^{4}} M^{3}
$$
Apres les calculs des constantes, on obtient
$$
\tau=10^{-16} M^{3} s \cdot k g^{-3}
$$
cst la durée de vie d'un trou noir une fois qu'il commence à s'évaporer. La température de Hawking d'un trou noir peut être approchée des valeurs des constantes
$$
T \cong \frac{10^{23}}{M}
$$
c'est seulement à propos de $10^{-17} K^{12}$ au-dessts du zéro absolu même pour les plus petits trous noirs stellaires (environ 3 masses solaires).




\section{ Lois de la thermodynamiques des trous noirs}

Bekenstein $(1972-1973)$ a remarque que l'aire de 1 'horizon $A_{h}$ a un
..............

 puisse s'identifier au terme de la quantite de chaleur fournie au systeme $\delta Q=$ T $\delta S$ ce qui nous pousse à identifier l'aire du trou noir à son entropie et la gravité de surface $\kappa$ à la température $T[22] .$ Puisque classiquement les trous noirs absorbent et n'émettent aucun rayonnement, leur temperature doit être nulle! Mais comme on va le voir. Hawking a montré avec une étude semi-classique que les trous noirs rayonnent et se comportent comme un corps noir.

On est done conduit à associer au trou noir une entropie $S$ proportionnelle a l'aire de 1 'horizon $A_{h}$ et une temperature $T$ proportionnelle a la gravite de surface $x:$
$$
S \propto A_{h} \quad T \propto \kappa
$$
Aussi l' equation( 3.2 ), Comme nous l'avons dejà enonce, marque le $1^{\circ}$ principe de la thermodynamique des trous noirs.
\subsection{Second principe : : Hawking areas theorem }
L'aire de l'horizon d'un trou noir, et donc son entropie ne peut pas décroitre
$$
\delta A_{h} \geq 0
$$
Mais une fois la mécanique quantique entre en jeu, on constate que par effet Hawking, le trou noir s'évapore (en perdant sa masse et en réduisant son aire), alors l'entropie diminuerait de nouveau par ce processus quantique. Toutefois, l'entropie emportée par les particules rayonnées par effet Hawking lors de l'evaporation correspond a une augmentation d'entropie nécessairement superieure a la diminution d'entropie du trou noir. Ainsi, la variation globale de l'entropie généralisee $S=S_{T N}+S_{e x t}$ va toujours dans le sens d'un accroissement $[8] .$
$$
\delta S \geq 0
$$
1. C'est une variation d' cnergie de radiation par effet Hawking

\subsection{Premier principe :  Relation différentielle entre la masse, la surface, la charge et le moment cinétique }


Pour un trou noir de Kerr-Newman, il est possible de dériver une équation analogue à la première loi de la thermodynamique Bekenstein $[1973] .$ Nous commençons par prendre la zone rationalisée $a$ d'un tel trou noir de masse $M,$  le moment cinétique $J$ et la charge $Q .$ On peut constater que :
$$
\begin{array}{c}
\alpha=r_{+}^{2}+a^{2} \\
\alpha=2 M r_{+}-Q^{2}
\end{array}
$$
where
$$
r_{\pm}=M \pm \sqrt{M^{2}-Q^{2}-a^{2}}
$$
$$
11
$$



sont des valeurs de la coordonnée radiale correspondant aux horizons des événements d'un trou noir de Kerr Newman.  On peut obtenir une formule pour la différence de masse $\delta M$ entre deux trous noirs de Kerr-Newman légèrement différents en faisant varier l'équation 1.35
$$
\delta \alpha=2\left(r_{+} \delta M+M \delta r_{+}-Q \delta Q\right)
$$
We now separately evaluate $\delta r_{+}:$
$$
\delta r_{+}=\delta M+\frac{2}{r_{+}-r_{-}}\left(M \delta M-a\left(\frac{\delta J}{M}+\frac{a}{M} \delta M-Q \delta Q\right)-Q \delta Q\right)
$$
By substituting this result into equation [1.37] we get:
$$
\delta \alpha=2\left(r_{+} \delta M+M \delta M+\frac{2}{r_{+}-r_{-}}\left(M^{2} \delta M-a \delta J+a^{2} \delta M-M Q \delta Q\right)-Q \delta Q\right)
$$
Our goal is to get a formula for the change of mass, To achieve it we will put all terms containing $\delta M$ to the left side:
$$
\begin{array}{c}
2\left(r_{+}+M+\frac{2}{r_{+}-r_{-}}\left(M^{2}+a^{2}\right)\right) \delta M= \\
=\delta \alpha+2 \frac{2}{r_{+}-r_{-}}\left(a \delta J+\left(M+\frac{r_{+}-r_{-}}{2}\right) Q \delta Q\right) \\
\frac{2}{r_{+}-r_{-}}\left(r_{+}\left(r_{+}-r_{-}\right)+M\left(r_{+}-r_{-}\right)+2 M^{2}+2 a^{2}\right) \delta M= \\
=\delta \alpha+\frac{4}{r_{+}-r_{-}}\left(a \delta J+r_{-} Q \delta Q\right) \\
\frac{2}{r_{+}-r_{+}}\left(2 M\left(r_{+}-r_{-}\right)+2 M^{2}-2 Q^{2}-2 a^{2}+2 M^{2}+2 a^{2}\right) \delta M= \\
=\delta \alpha+\frac{4}{r_{+}-r_{-}}\left(a \delta J+r_{+} Q \delta Q\right)
\end{array}
$$
Using now equation [1.35] to express $M_{+}$ we will arrive to the final form of the first law:
$$
\delta M=\frac{r_{+}-r_{-}}{4 \alpha} \delta_{\alpha}+\frac{a}{\alpha} \delta J+\frac{r_{+} Q}{\alpha} Q \delta Q
$$
If we now compare equation [1.40] with its thermodynamical analogue, we see that if the change in black hole's mass corresponds to the change in energy, then the term $(a / \alpha) \delta J+\left(r_{+} Q / \alpha\right) Q \delta Q$ is equivalent to the work done on the black hole in order to change its charge and angular momentum. Specifically, a / a represents the rotational angular frequency of the black hole and $\left(r_{+} Q / \alpha\right) Q$ is the electric potential on the outer event horizon. Since $\delta \alpha$ is proportional to the black hole's entropy, the term $\left(r_{+}-r_{-}\right) / 4 \alpha$ (multiplied by a certain constant) is analogous to black hole temperature.

Temperature defined in this way is non-negative, just like the thermodynamic temperature, and equals zero only for extremal black holes. However, the term temperature can not be successfully applied to a black hole fully described by general relativity. because such a black hole does not radiate and thus ean not be in thermal equilibrium with black body radiation at any non-zero temperature. Therefore, in the context of elassical general relativity, this analogy should be regarded as purely formal.



Dans leur article Bardeen et autres (1973), Bardeen, Carter et Hawking ont dérivé la première loi de la dynamique des trous noirs d'une manière plus générale qui s'applique à tout espace-temps stationnaire axisymétrique asymptotiquement plat qui comprend un trou noir. Ils ont présenté la forme intégrale et différentielle de la première loi. Nous dériverons les deux versions en détail en suivant l'approche utilisée dans l'article original.


 Bardeen-Carter-Hawking Formulation

 Apendix A


La variation $\delta M$ de la masse $d^{\prime} u n$ trou noir. entraine une variation de l'energie potentiel élec-
rayonnements$^{1} \frac{x}{8 \pi} \delta A_{n}$
Le première principe se traduit donc par l'expression( 3.2 ) de Bekenstein :
$$
\delta M=\frac{\kappa}{8 \pi} \delta A_{h}+\Omega_{H} \delta I+\Phi_{H} \delta Q
$$


\subsection{Principe zéro : la constante de la gravité de la surface à l'horizon }

 
La gravite de surface $\mathrm{k}$ d'un trou noir stationnaire est constante sur l'horizon du trou noir.
En présence de la matiere on peut prouver cette loi en utilisant les equations d'Einstein avec la condition dite d'energie dominante pour le tenseur d' energie-impulsion[Bardeen( 1973$)][19]$.




\subsection{Troisième principe :  la gravité de la surface n'est jamais annulée }


Le troisieme principe de la thermodynamique ordinaire, proposé par Nernst en $1906,$ s' énonce de la maniere suivante :

II est impossible de réduire la temperature de $n$ 'importe qu "il système à zéro absolue par un nombre funi d'opérations.
Bardeen (1973) a formulé l'analogue du troisième principe pour les trous noirs dans la forme suivante:

Il est impossible par aucune procédure de réduire la gravite de surface d'un trou noir à $\kappa=0$ par un nombre fini d'operations/21.

Wald en 1974 a confirme la formulation de Bardeen, et il a montré qu'il est difficile de se rapprocher d'un trou noir extrémale de $x=0 .$ Cependant, La version de Planck du troisième principe, indiquant que l'entropie d'un système tend vers zéro lorsque la température se rapproche de $z$ iro absolue, est violée par un trou noir extremal $(x=0): 1$ 'aire de 1 'horizon d'un trou noir de Kerr-Newman ne s'annule pas lorsque $\kappa \rightarrow 0[2][19] .$ Le troisième principe correspond donc al limpossibilite d'obtenir un trou noir extrémal : $M^{2}=a^{2}+Q^{2}$ [13]. L'analogie mathématique entre les lois thermodynamiques des trous noirs et les lois de la thermodynamique ordinaire définies ci-dessus sont affichées dans le tableau(3.1).

TABLE $3.1-L$ 'analogie entre la thermodynamique standard et la thermodynamique des trous noirs.

 \subsection{ the area increase theorem }

 \subsection{ black hole entropy}



 \subsection{ Generalized second law}





\section{L’analogie entre la thermodynamique standard et la thermodynamique des trous noirs }


 
\begin{table}
	{ \renewcommand{\arraystretch}{1.4}
\begin{tabular}{|l|l|l|}
\hline
 Pincipe & Thermodynamique Standard  & Thermodynamique des trous noirs\\
 \hline
 Principe zéro & La température T d’un corps est &  La gravité de surface k d’un trou \\
            &             la même  partout  dans celui-ci &  noir stationnaire est constante  \\
             &              à l’équilibre thermique.  &  sur tout l’horizon des événements \\
\hline
Premier principe & $dE=T\delta S+ \delta W$ . & $dM=\dfrac{k}{8\pi}\delta A+\Omega_{h}\delta J+\Phi_{h}\delta Q$ \\
\hline
Deuxième principe & L'entropie d'un système isolé ne   & L'air A de l'horizon des événements \\
       &  peut qu'augmenter $\delta S \geq 0$.   &  de chaque trou noir ne peut pas  \\
         &     & décroître $\delta A\geq 0$.\\
\hline
Troisième principe & On ne peut atteindre $T = 0$ & On ne peut pas atteindre $k = 0$\\
               &      par aucun processus physique.   &  par aucun processus. \\
\hline
\end{tabular}}
\caption{L'analogie entre la thermodynamique standard et la thermodynamique des trousnoirs.}
\end{table}









\begin{eqnarray}
\textrm{Enthalpie}\quad  H &\leftrightarrow&  M\quad  \textrm{Masse}   \nonumber\\
\textrm{Temperature}\quad  T &\leftrightarrow&  \frac{\kappa}{2\pi} \quad  \textrm{Surface Gravity}   \nonumber\\
\textrm{Entropy}\quad  S &\leftrightarrow&  \frac{A}{4 } \quad  \textrm{Horizon Area}   \nonumber
\label{eq:01}
\end{eqnarray}

in units where $G=c=\hbar =k_B=1$,



\be\label{A1}
\begin{array}{|l|c|l|c|}
\hline
\multicolumn{2}{|c|}{\mbox{Thermodynamics}} & \multicolumn{2}{|c|}{\mbox{Black Hole Physics}} \\
\hline
\mbox{Enthalpy} & H & \mbox{Mass} & M\\
\hline
\mbox{Temperature} & T & \mbox{Surface Gravity} & \frac{  \kappa}{2\pi}\\
\hline
\mbox{Entropy}  &S &\mbox{Horizon Area} & \frac{A}{4} \\
\hline
\mbox{Pressure}  &P & \mbox{Cosmological Constant}  &  -\frac{\Lambda}{8\pi} \\
\hline
\mbox{First Law}  &dH= T dS +VdP + \ldots  & \textrm{First Law} & dM= \frac{\kappa}{8\pi} dA  +VdP + \ldots\\
\hline
\end{array} 
\ee


 
 
 
 
 \subsection{ la Volume thermodynamique et la pression}
 En se concentrant entièrement sur les trous noirs asymptotiquement plats et AdS, nous considérons la proposition que la pression thermodynamique est donnée par 
 \be\label{PLambda}
P = - \frac{1}{8 \pi} \Lambda=\frac{(d-1)(d-2)}{16 \pi l^2}\,,
\ee

où $d$ représente le nombre de dimensions de l'espace-temps, $\Lambda$ est la constante cosmologique et $l$ désigne le rayon AdS.  
La casse asymptotiquement plate a $P=0$.


\begin{equation}
T= \left(\frac{\partial H}{\partial S} \right)_P;,\\
V= \left(\frac{\partial H}{\partial P} \right)_S.
\end{equation}




 
Une fois le volume thermodynamique connu et la constante cosmologique identifiée avec la pression thermodynamique, on peut, pour un trou noir donné, noter l'équation d'état du "fluide" correspondant en relation avec la pression, la température, le volume et les autres paramètres externes caractérisant le trou noir,
$P=P(V,T,J_i,Q)$

Similaire aux références
 
 \cite{KubiznakMann:2012, GunasekaranEtal:2012},

avec $V$ étant le volume thermodynamique




\subsection{ Les transition des phases et thermodynamique equilibre }


La thermodynamique d'équilibre est régie par l'énergie libre de Gibbs $G=G(T,P,J_i, Q)$, dont le minimum global donne l'état d'un système pour une valeur fixe.1 Puisque la masse du trou noir est interprétée comme l'enthalpie, nous avons la relation thermodynamique suivante :
\be
G=M-TS=G(P,T,J_1,\dots,J_N, Q)\,. 
\ee

Ici, $T$ et $S$ représentent la température de l'horizon et l'entropie du trou noir.

On obtient ainsi des diagrammes $G-T$ qui intègrent les informations sur les éventuelles transitions de phase
Une proposition alternative intéressante qui a attiré l'attention ces dernières années est d'envisager une "géométrie thermodynamique efficace" et d'étudier ses singularités de courbure . Ces singularités fournissent ensuite une information sur la présence de points critiques et donc sur les éventuelles transitions de phase dans l'espace-temps donné

La stabilité thermodynamique locale d'un ensemble canonique est caractérisée par la positivité de la chaleur spécifique à pression constante


\be\label{CP}
C_P\equiv C_{P,J_1,\dots,J_N,Q}=T\left(\frac{\partial S}{\partial T}\right)_{P,J_1,\dots,J_N,Q}\,.
\ee

Nous prenons la négativité de $C_P$ comme un signe d'instabilité thermodynamique locale.

