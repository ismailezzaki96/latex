%\section{Columns}
% --------------------------------------------------- Slide --
%\subsection{Columns}
\label{columns}
\begin{frame}\frametitle{Demonstration of Clifford’s rule }

page 6



% 
%  \textbf{Gressman} : “a product of two extensives quantities u and v is defined as extensive quantity or a scalar.”
%  
% clifford consider a n-fundamental units (e1,e2,...,en) 
% 
% such that n is an integer ,  all$ (ej)^2= +e or -e $ and $ej.ek=-ek.ej$ $(j#k)=$\\
% where e denotes the identity. all product of linear factors must contain either basic terms of odd order (ej, ejekel, ... such that j#k#l), or basic terms of even order ( 1, ej ek, ...such that j# k). There can be no such term of higher order than the n. and since we have altogether one term of order zero, n of order 1, (½ (n - 1); n odd,n/2; even) of order 2,... , one of order n; the total number of basic terms is 2^n units.
% 
% He uses the selective symbols Vo, VI, ..., Vn to denote the subspaces     containing expressions of order 0, 1, ..., n.
% 
%  

\end{frame}
