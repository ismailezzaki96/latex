%\section{Columns}
% --------------------------------------------------- Slide --
%\subsection{Columns}
\label{columns}
\begin{frame}\frametitle{Demonstration of Clifford’s rule }

\begin{itemize}
	\item 
Clifford apply a general definition of product given by Gressman : “\textbf{a product of two extensives quantities u and v is defined as extensive quantity or a scalar}.”


	\item 

Clifford consider a n-fundamental units $(e_1,e_2,...,e_n)$ \\

\begin{center}
	$(e_j)^2=+1$or $-1$  and $e_j.e_k=-e_k.e_j (j \neq k)$ 
\end{center}
	\item 
%where e denotes the identity. \\
All product of linear factors : \\odd order $(ej, e_j e_k e_l, ... $such that $j \neq k \neq l)$,
\\or  even order $( 1, e_j e_k, ...$ such that $j \neq k)$. \\

%There can be no such term of higher order than the n. 

	\item 
The total number of basic terms is $2^n$ units.


\end{itemize}

\end{frame}


\begin{frame}\frametitle{Demonstration of Clifford’s rule }

He writes $w$ for the product of all $n-$units, $w \equiv e_1 e_2...e_n$ and enquires into the value of its square:
$w^2 \equiv e_1 e_2....e_n e_1 e_2...e_n$,\\
which after $r-$interchanges becomes:   

$w^2 \equiv  (-1)^r (e_{1})^2 (e_{2})^2 ...(e_{n})^2  \equiv (-1)^r if ()e_j)^2=1 (-1)^{r+n} if ()e_j)^2=-1$

\end{frame}



\begin{frame}\frametitle{Demonstration of Clifford’s rule }

 Next, he considers the nature of the multiplication w ej' He observes that:\\
 
\[

   e_j w = 
\begin{cases}
+ e_1 e_2 e_3 ...e_n,& \text{if } (e_j)^2 = + 1\\
- e_1 e_2 e_3 ...e_n,              &  \text{if } (e_j)^2 = - 1
\end{cases}

\\ 
\text{and }
\\
    w e_j= 
\begin{cases}
(1)^n e_1 e_2 e_3 ...e_n,& \text{if } (e_j)^2 = + 1\\
(1)^{n-1} e_1 e_2 e_3 ...e_n,              &  \text{if } (e_j)^2 = - 1
\end{cases}

\]
 
for both cases $(e_j)^2 = \pm 1$, $w e_j = \pm e_j w$ according as n is odd or even .\\
 
 
 $w e_j $ is commutative or anti-commutative 
 

Therefore, we can put :\\

$w \equiv \sqrt{-1}$ 
\end{frame}


\begin{frame}\frametitle{Demonstration of Clifford’s rule }
4\\



he considers the factors of units of even order

\begin{center}
	$1, e_j e_k, e_j e_k e_i e_m...$such that $j \neq k \neq I$;
\end{center}
they form an algebra by themselves which he calls the even subalgebra \\

the even subalgebra with basic elements 
\begin{center}
	$\{I; e_2 e_3; e_l e_3; e_l e_2\}$
\end{center}
 gives the quaternions with $i == e_2 e_3$, $j == e_l e_3$, $k == e_1 e_2;$





\end{frame}
\begin{frame}\frametitle{Demonstration of Clifford’s rule }

The general element of the even subalgebra can be expressed as :\\


where $q$ and $q'$ are quaternions. \\

quadratic form :\\




we get Clifford's rule \\
{\color{red}
\textbf{\begin{center}
	$e_j e_k + e_k e_j = 2 \delta_{jk} 1 \;\;,\;\; (j  ,k=I, ... ,n).$
\end{center}}}


\end{frame}