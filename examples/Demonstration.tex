%\section{Columns}
% --------------------------------------------------- Slide --
%\subsection{Columns}
\label{columns}
\begin{frame}\frametitle{Demonstration of Clifford’s rule }

Clifford apply a general definition of product given by Gressman : “a product of two extensives quantities u and v is defined as extensive quantity or a scalar.”

1\\

clifford consider a n-fundamental units $(e_1,e_2,...,e_n)$ \\

all $(e_j)^2=+1$or $-1$  and $e_j.e_k=-e_k.e_j (j \neq k)$ where e denotes the identity. 
all product of linear factors must contain either basic terms of odd order $(ej, e_j e_k e_l, ... $such that $j \neq k \neq l)$,
or basic terms of even order $( 1, e_j e_k, ...$ such that $j \neq k)$. \\

There can be no such term of higher order than the n. 

the total number of basic terms is $2^n$ units.




\end{frame}


\begin{frame}\frametitle{Demonstration of Clifford’s rule }
2\\

he writes $w$ for the product of all $n-$units, $w \equiv e_1 e_2...e_n$ and enquires into the value of its square:
$w^2 \equiv e_1 e_2....e_n e_1 e_2...e_n$,\\
which after $r-$interchanges becomes:   

$w^2 \equiv  (-1)^r (e_{1})^2 (e_{2})^2 ...(e_{n})^2  \equiv (-1)^r if ()e_j)^2=1 (-1)^{r+n} if ()e_j)^2=-1$

\end{frame}



\begin{frame}\frametitle{Demonstration of Clifford’s rule }
3\\

 Next, he considers the nature of the multiplication w ej' He observes that:\\
 
relation 8 ansd 9 
 
for both cases $(e_j)^2 = \pm 1$, $w e_j = \pm e_j w$ according as n is odd or even .\\
 
 
 $w e_j $ is commutative or anti-commutative 
 

Therefore, we can put :\\

$w \equiv \sqrt{-1}$ 
\end{frame}


\begin{frame}\frametitle{Demonstration of Clifford’s rule }
4\\



he considers the factors of units of even order

\begin{center}
	$1, e_j e_k, e_j e_k e_i e_m...$such that $j \neq k \neq I$;
\end{center}
they form an algebra by themselves which he calls the even subalgebra \\

the even subalgebra with basic elements 
\begin{center}
	$\{I; e_2 e_3; e_l e_3; e_l e_2\}$
\end{center}
 gives the quaternions with $i == e_2 e_3$, $j == e_l e_3$, $k == e_1 e_2;$





\end{frame}
\begin{frame}\frametitle{Demonstration of Clifford’s rule }

The general element of the even subalgebra can be expressed as :\\


where $q$ and $q'$ are quaternions. \\

quadratic form :\\




we get Clifford's rule \\
$e_j e_k + e_k e_j = 2 Dirac_{jk} 1 , (j,k=I, ... ,n).$


\end{frame}