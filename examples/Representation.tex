%\section{Definition}
% --------------------------------------------------- Slide --
%\subsection{Definition}
\label{definition}
\begin{frame}\frametitle{Dirac representation}
$\gamma ^{0}={\begin{pmatrix}I_{2}&0\\0&-I_{2}\end{pmatrix}},\quad \gamma ^{k}={\begin{pmatrix}0&\sigma ^{k}\\-\sigma ^{k}&0\end{pmatrix}},\quad \gamma ^{5}={\begin{pmatrix}0&I_{2}\\I_{2}&0\end{pmatrix}}$
Dirac representation
 
The gamma matrices we have written so far are appropriate for acting on Dirac spinors written in the Dirac basis; in fact, the Dirac basis is defined by these matrices. To summarize, in the Dirac basis:


 
\end{frame}
\begin{frame}\frametitle{Majorana representation}
${\displaystyle {\begin{aligned}\gamma ^{0}&={\begin{pmatrix}0&\sigma ^{2}\\\sigma ^{2}&0\end{pmatrix}},&\gamma ^{1}&={\begin{pmatrix}i\sigma ^{3}&0\\0&i\sigma ^{3}\end{pmatrix}},&\gamma ^{2}&={\begin{pmatrix}0&-\sigma ^{2}\\\sigma ^{2}&0\end{pmatrix}},\\\gamma ^{3}&={\begin{pmatrix}-i\sigma ^{1}&0\\0&-i\sigma ^{1}\end{pmatrix}},&\gamma ^{5}&={\begin{pmatrix}\sigma ^{2}&0\\0&-\sigma ^{2}\end{pmatrix}},&C&={\begin{pmatrix}0&-i\sigma ^{2}\\-i\sigma ^{2}&0\end{pmatrix}},\end{aligned}}}$
\\

 (The reason for making all gamma matrices imaginary is solely to obtain the particle physics metric $(+, −, −, −)$, in which squared masses are positive. The Majorana representation, however, is real. One can factor out the $i$ to obtain a different representation with four component real spinors and real gamma matrices. The consequence of removing the $i$ is that the only possible metric with real gamma matrices is $(−, +, +, +)$.)

\end{frame}
\begin{frame}\frametitle{ Weyl (chiral) representation }
Another common choice is the Weyl or chiral basis,[5] in which ${\displaystyle \gamma ^{k}}\gamma ^{k}$ remains the same but ${\displaystyle \gamma ^{0}}\gamma ^{0}$ is different, and so ${\displaystyle \gamma ^{5}}\gamma ^{5}$ is also different, and diagonal,


$\gamma ^{0}={\begin{pmatrix}0&I_{2}\\I_{2}&0\end{pmatrix}},\quad \gamma ^{k}={\begin{pmatrix}0&\sigma ^{k}\\-\sigma ^{k}&0\end{pmatrix}},\quad \gamma ^{5}={\begin{pmatrix}-I_{2}&0\\0&I_{2}\end{pmatrix}},$


${\displaystyle \psi _{L}={\frac {1}{2}}\left(1-\gamma ^{5}\right)\psi ={\begin{pmatrix}I_{2}&0\\0&0\end{pmatrix}}\psi ,\quad \psi _{R}={\frac {1}{2}}\left(1+\gamma ^{5}\right)\psi ={\begin{pmatrix}0&0\\0&I_{2}\end{pmatrix}}\psi .}$

The Weyl basis has the advantage that its chiral projections take a simple form,




\end{frame}
