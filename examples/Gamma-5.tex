%\section{Description}
% --------------------------------------------------- Slide --
%\subsection{Description}
\label{The fifth Gamma matrix}
\begin{frame}\frametitle{The fifth Gamma matrix}

Define an additional γ-matrix by:\\

$\gamma^5\equiv i/4! \varepsilon_{\nu \rho \sigma  \eta}  \gamma^\nu \gamma^\rho \gamma^\sigma \gamma^\eta$\\
$\gamma^5\equiv i\gamma^0\gamma^1\gamma^2\gamma^3$\\

Note:
$(\gamma^5)^2 = 1$
anti-commutes with every other $\gamma$:
$\{\gamma^5,\gamma^\mu\}=0 \Longrightarrow \gamma^5\times \gamma^\mu = - \gamma^\mu \times \gamma^5 $


\end{frame}



\label{The fifth Gamma matrix}

\begin{frame}\frametitle{The fifth Gamma matrix}



$\overline{\Psi} \gamma^5 \Psi \Longrightarrow (P\Psi)^{+} \gamma^0 \gamma^5  (P\Psi)  \Longrightarrow \Psi^{+} (\gamma^0 )^{+} \gamma^5 \gamma^0 \Psi \Longrightarrow - \Psi^{+}  (\gamma^0 )^{+}  \gamma^5  \Psi \Longrightarrow  - \overline{\Psi} \gamma^5 \Psi \Longrightarrow $ Pseudo-scalar 
 



To describe the parity-violating in weak interaction, we could (and will!) mix vector and axial interactions.

$(\overline{\Psi}  \gamma^\mu \Psi) \pm (\overline{\Psi}  \gamma^\mu \gamma^5 \Psi)$




\end{frame}
