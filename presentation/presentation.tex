%===============================================================================
% Template Name:      SUnORE Starter Presentation template
% Template URI:       http://sunore.co.za/sunore-presentation/
% Description:        Starter Presentation template for SUnORE 
%                     Department of Industrial Engineering, 
%                     Stellenbosch University
% Version:            1.1.0
% Author:             Johan Janse van Rensburg
% Author URI:         http://johanjvrens.co.za/
% License:            MIT License
% License URI:        http://opensource.org/licenses/MIT
%===============================================================================
\documentclass[11pt,aspectratio=169]{beamer}


\usetheme{auriga} %Themes http://www.hartwork.org/beamer-theme-matrix/
%\usecolortheme{auriga}

\hypersetup{pdfpagemode=FullScreen}


\definecolor{red}{RGB}{181, 23, 0}
\definecolor{blue}{RGB}{0, 118, 186}
\definecolor{gray}{RGB}{146, 146, 146}
\definecolor{colorA}{RGB}{96, 34, 59}
\definecolor{colorB}{RGB}{140, 151, 154}
\definecolor{secinhead}{RGB}{249,196,95}
\definecolor{titlebg}{RGB}{51,51,51}
%\setbeamercolor{structure}{fg=colorA,bg=colorB}
\setbeamercolor{secsubsec}{fg=secinhead,bg=black}
%\setbeamercolor{frametitle}{fg=secinhead,bg=gray}

%=================================================
% packages and new commands
%=================================================
\usepackage[ruled, linesnumbered, vlined]{algorithm2e}
\usepackage{multirow, algorithmic, amsmath}
\usepackage[francais]{babel}


\usepackage{pgfpages}
\usepackage{fancyvrb}
\usepackage{tikz}
\usepackage{pgfplots}

% \ifnotes
% \setbeamertemplate{note page}[plain]
% \setbeameroption{show notes on second screen=right}
% \fi


\makeatletter
\let\insertsupervisor\relax
\newcommand\supervisortitle{Sous la direction de}
\mode<all>
{
	\newcommand\supervisor[1]{\def\insertsupervisor{#1}}
	\titlegraphic{}
}
\defbeamertemplate*{title page}{supdefault}[1][]
{
	\vbox{}
	\vfill
	\begingroup
	\centering
	\begin{beamercolorbox}[sep=8pt,center,#1]{title}
		\usebeamerfont{title}\inserttitle\par%
		\ifx\insertsubtitle\@empty\relax%
		\else%
		\vskip0.25em%
		{\usebeamerfont{subtitle}\usebeamercolor[fg]{subtitle}\insertsubtitle\par}%
		\fi%     
	\end{beamercolorbox}%
	\vskip1em\par
	\begin{beamercolorbox}[sep=8pt,center,#1]{author}
		\usebeamerfont{author}\insertauthor
	\end{beamercolorbox}
	\ifx\insertsupervisor\relax\relax\else
	\begin{beamercolorbox}[sep=8pt,center,#1]{}
		\usebeamerfont{author}\supervisortitle:~\insertsupervisor
	\end{beamercolorbox}\fi
	\begin{beamercolorbox}[sep=8pt,center,#1]{institute}
		\usebeamerfont{institute}\insertinstitute
	\end{beamercolorbox}
	\begin{beamercolorbox}[sep=8pt,center,#1]{date}
		\usebeamerfont{date}\insertdate
	\end{beamercolorbox}\vskip0.5em
	
	{\usebeamercolor[fg]{titlegraphic}\inserttitlegraphic\par}
	\endgroup
	\vfill
}










\title[ \hspace{0.8cm} \insertframenumber/\inserttotalframenumber]{{\sc Physique des trous noirs: thermodynamique et transitions de phase }}


 
\author{{ Ismail EZZAKI}}

\supervisor{{ Mohamed \textsc{CHABAB}}}
\date  {\today}
\institute  { Université Cadi Ayyad \\	Faculté des Sciences Semlalia\\	Laboratoire de Physique des Hautes Énergies et Astrophysique}





%=================================================
% start presentation
%=================================================
\begin{document}
	
	\setbeamertemplate{titlepage}[supdefault]

{
  % rather than use the frame options [noframenumbering,plain], we make the
  % color match, so that the indicated page numbers match PDF page numbers
  \setbeamercolor{page number in head/foot}{fg=background canvas.bg}
  \begin{frame}
    \titlepage
  \end{frame}
}
\begin{frame}

  \frametitle{Outline}
	\tableofcontents
\end{frame}

%========================
% your slides:
%========================
\section{Introduction}

\section{Les Trous Noirs En Astrophysique}
\subsection{Formation des Trous Noirs}
\subsection{Les differents types de Trous Noirs}
\subsection{Détection des Trous Noirs }
\subsection{Propriétés des trous noirs}
\section{ les Trous Noirs En Relativité générale}

\subsection{Les équations d'Einstein}
\subsection{Le trou noir dans un espace asymptotiquement plat}

\section{ LA THERMODYNAMIQUE DES TROUS	NOIRS}

\subsection{Les quatre lois de la thremodynamique des trous noirs}

\subsection{Rayonnement du trou noir}

\subsection{Thermodynamique du trou noir  AdS}

	
\subsection{Géométrie thermodynamique et transition de phase}

\input{slides/bullets}
\input{slides/split}
\input{slides/figure}
\input{slides/centered}
\input{slides/monospace}
\input{slides/brackets}
\input{slides/link}

\SetKwInOut{Input}{Input}\SetKwInOut{Output}{Output}
\begin{frame}\frametitle{Algebraic structure}

An algebraic structure consists of
a set of elements B 
binary operators ${+, \times}$
and a unary operator ${1‘ }$
Such that following holds :
$\forall a,b \in B $\\
\textit{Closure}:\begin{center}

$a+b \in  B$	and	$a\times b \in B$ \end{center}


\textit{Commutativity}: \begin{center}

$a+b = b+a$	and	$a\times b = b\times a$
\end{center}

\textit{Associativity}: 
\begin{center}
 

$a+(b+c)=(a+b)+c$ and	$a\times (b\times c) = (a\times b)\times c$ 
\end{center}


\textit{Identity}: 
\begin{center}

$a+0 =  a$		and	$a\times 1=a$ 
\end{center}

\end{frame}

%\section{Blocks}
% --------------------------------------------------- Slide --
%\subsection{Blocks}
\label{blocks}
\begin{frame}\frametitle{Historical events}
 \begin{itemize} 
\item
\textbf{500 B.C.}: real numbers and Synthetic Geometry by Euclid
\pause  \item
\textbf{1545}:  The Italian mathematician Gerolamo Cardano has introduced the complex numbers .
\pause 
\item
\textbf{1843}: Hamilton invents Quaternions and the rules for operation on them.
\pause 
\item
\textbf{1862} :  The second edition of Grassmann's theory of extension ,led to the development of Clifford Algebras , vector analysis…
\pause 
\item
\textbf{1878}: Clifford introduces “geometric algebra”, but dies at age 34. 
\pause 
\item
\textbf{1881-1884} :Gibbs developed a three-dimensional vector analysis based on Grassmann's and Hamilton's ideas.
\pause 
\item
\textbf{1920}: Renaissance in quantum mechanics (Pauli, Dirac)
\end{itemize}

\end{frame}

%\section{Definition}
% --------------------------------------------------- Slide --
%\subsection{Definition}
\label{definition}
\begin{frame}\frametitle{Quaternions}
Quaternions are an extension of complex numbers with three imaginary units $i$, $j$, and $k$.

Quaternion algebra is an associative, non-commutative division ring with four basic elements: a real identity 1 and three different kinds of imaginary units i, j, and k called pure quaternion basis which obey the multiplication rules:
\begin{center}

$ i^2 =j^2 = k^2 = -1$\\ $i\times j = -j\times i = k$ \\ $i \times j \times k=1 $
\end{center}
The quaternion operator written in the conventional form :
\begin{center}
 $q=q_{0}\times 1+q_{1}\times i+q_{2}\times j+q_{3}\times k$

\end{center}

with $q_{0},q_{1},q_{2},q_{3}$ are Cayley-Klein parameters.

\end{frame}


%\section{Hyperlinks}
% --------------------------------------------------- Slide --
%\subsection{Hyperlinks Code}
\label{hyperlinks}
\begin{frame}\frametitle{Exemple of Clifford Algebras}



Most used Cliffords Algebra in physics :\\
\begin{center}
Nondegenerate Clifford algebra  $Cl_{p,q}$
\end{center}


\begin{table}[htp]
 \centering
\begin{tabular}{ccc}\hline

\textbf{Notation} & \textbf{Geometry} & \textbf{Dim.}\\\hline
$Cl_{2,0}$ & plane & 4\\\hline
$Cl_{0,1}$ & complex & 2\\\hline
$Cl_{0,2}$ & quaternios & 4\\\hline
$Cl_{0,3}$ & Pauli & 8\\\hline
$Cl_{1,3}$ & Dirac & 16\\\hline
$Cl_{m,n}$ & sigature $(m,n)$ & $2^{n+m}$\\\hline
\end{tabular}

 \caption{Exemple of Clifford Algebras  $Cl_{p,q}$ }
\end{table}




\end{frame}

%\subsection{Clifford algebra simple example: $Cl(3,0)$ \& $Cl(1,3)$ }
%%\section{Example}
% --------------------------------------------------- Slide --
%\subsection{Example}
\label{The space-time algebra}
\begin{frame}\frametitle{The space-time algebra}
GA in 4D with Minkowski-Metric$ (+,-,-,-)$

\end{frame}

% \subsection{Some applications of Clifford algebra}
%\section{Figures}
% --------------------------------------------------- Slide --
%\subsection{Figures}
\label{applications}
\begin{frame}\frametitle{applications}

The formalism of Clifford algebras allows wide applications, in particular,

\begin{itemize}
 \item 
 Mechanics:  Foucault pendulum
 \item 

Electro-magnetostatics
 \item 

Dispersion and diffraction E.M.
 \item 

Quantum Mechanics: spin precession 
 \item 

Field Theory: Dirac equation 
 \item 

General Relativity
\end{itemize}


\end{frame}

% \section{ Gamma Matrices}
%\subsection{Dirac equation}
%%\section{Lists}
% --------------------------------------------------- Slide --
 
% --------------------------------------------------- Slide --
%\subsection{Pause}
\label{pause}
\begin{frame}\frametitle{Lists - Itemize with Pause}
  \begin{itemize}
    %\pause \item Point A
    %\pause \item Point B
    \begin{itemize}
    I think that thisslide will be removed, has to be removed
      %\pause \item part 1
      %\pause \item part 2
    \end{itemize}
    %\pause \item Point C
   % \pause \item Point D
  \end{itemize}
\end{frame}

 

% \subsection{Gamma matrices’ rule}
\begin{frame}[allowframebreaks]
 \frametitle{Gamma matrices' rule}
\begin{itemize}
\item The klein Gordon equation of a free massive particle :
  \begin{center}
    $P^\mu P_\mu \Psi = m^2 \Psi$, \qquad  $(\mu=0,1,2,3)$
  \end{center}
  % Where the contravariant and covariant energy-momentum operators are given by
  Where :
  \begin{center}
    $P^\mu = \imath \hbar g^{\mu\nu} \partial_\nu $  \qquad; \quad $P_\mu = \imath \hbar \partial_\mu$
  \end{center}
  And :
  \begin{center}
    $g^{\mu\nu}= diag(1,-1,-1,-1)$   
  \end{center}
  \item The linear matrix wave equation :
  \begin{center}
    $\gamma_\mu P^\mu \Psi= m\Psi$   \qquad (2)
  \end{center}
  with:
  \begin{center}
    $\gamma^\mu P^\mu = \imath (\gamma^0 \frac{\partial}{\partial t} +\gamma \nabla)$\\
    %\gamma is a 3-dimension matrix vector with components gamma1,2,3 
  \end{center}
  the equation (2) satisfy the equation (1) if and only if :
  \begin{center}
    $ (\gamma_\mu P^\mu)^2=P^\mu P_\mu I$  \\
  \end{center}
  \begin{center}
    $\Longrightarrow P^\mu P^\nu \dfrac{1}{2}(\gamma_\mu \gamma_\nu +\gamma_\nu \gamma_\mu )=P^\mu P^\nu g_{\mu\nu} I$
  \end{center}
  Equivalently if and only if Clifford's rules are satisfied :
  \begin{center}
  \item  $(\gamma_\mu \gamma_\nu +\gamma_\nu \gamma_\mu )=2 g_{\mu\nu} I$
  \end{center}
  %Therefore, the Dirac 1-matrices which brought in the spin of the electron turn out to be representations of elements of a Clifford algebra associated with Minkowski space.
\end{itemize}
\end{frame}

% \subsection{Gammma matrices' Properties}
%%\section{Tables}
% --------------------------------------------------- Slide --
%\subsection{Tables}
\label{tables}
\begin{frame}[allowframebreaks]\frametitle{ Gamma matrices' properties}
\begin{itemize}
\item  Gamma matrices {} must satisfy the basic commutation realtion:\\
    \begin{center}
     $[\gamma_\mu , \gamma_\nu]_+= 2g_{\mu\nu}$ \\
     \end{center}
     \begin{center}
\\ $\gamma_0^{2}=1 \qquad ; \gamma_k^2=-1 \quad ; \gamma_\mu .\gamma_\nu = 0 \quad (\mu \ne \nu) $
\end{center}
\begin{center}
\\$\gamma_0 \gamma_\mu^{\dagger}\gamma_0=\gamma_\mu \qquad$
\end{center}
\begin{center}
$ \sigma_{\mu\nu}= i\dfrac{1}{2}[\gamma_\mu,\gamma_\nu]$
\end{center}
\end{itemize}
\begin{center}
\[
  $\gamma_{0}$=\left(\begin{array}{cc}
    \sigma_0 & 0 \\
    0 & \sigma_0
    \end{array}\right) ~;  \qquad
$\gamma_{i}$=\left(\begin{array}{cc}
    0 & \sigma_i   \\
   -\sigma_i & 0
    \end{array}\right)    
    \]
\end{center}
\begin{center}
\end{center}
\begin{center}
\end{center}
\begin{center}
\end{center}
\begin{center}
\end{center}
\begin{center}
\end{center}
\begin{center}
\end{center}
\begin{center}
\end{center}
\end{frame}

% \subsection{The fifth Gamma matrix $\gamma^5$}
%%\section{Description}
% --------------------------------------------------- Slide --
%\subsection{Description}
\label{The fifth Gamma matrix}
\begin{frame}\frametitle{The fifth Gamma matrix}

Define an additional γ-matrix by:\\

$\gamma^5\equiv i/4! \varepsilon_{\nu \rho \sigma  \eta}  \gamma^\nu \gamma^\rho \gamma^\sigma \gamma^\eta$\\
$\gamma^5\equiv i\gamma^0\gamma^1\gamma^2\gamma^3$\\

Note:
$(\gamma^5)^2 = 1$
anti-commutes with every other $\gamma$:
$\{\gamma^5,\gamma^\mu\}=0 \Longrightarrow \gamma^5\times \gamma^\mu = - \gamma^\mu \times \gamma^5 $


\end{frame}



\label{The fifth Gamma matrix}

\begin{frame}\frametitle{The fifth Gamma matrix}



$\overline{\Psi} \gamma^5 \Psi \Longrightarrow (P\Psi)^{+} \gamma^0 \gamma^5  (P\Psi)  \Longrightarrow \Psi^{+} (\gamma^0 )^{+} \gamma^5 \gamma^0 \Psi \Longrightarrow - \Psi^{+}  (\gamma^0 )^{+}  \gamma^5  \Psi \Longrightarrow  - \overline{\Psi} \gamma^5 \Psi \Longrightarrow $ Pseudo-scalar 
 



To describe the parity-violating in weak interaction, we could (and will!) mix vector and axial interactions.

$(\overline{\Psi}  \gamma^\mu \Psi) \pm (\overline{\Psi}  \gamma^\mu \gamma^5 \Psi)$




\end{frame}

% \subsection{Bilinear covariants}
%\section{Theorem}
% --------------------------------------------------- Slide --
%\subsection{Theorem Code}
\label{theoremCode}
\begin{frame}\frametitle{Bilinear covariants}
\begin{itemize}
	\item 

All spinor fields live in the a 4-dimensional spacetime \\
spacetime time algebra
\begin{center}
\begin{tabular}{|c|c|c|c|}\hline
\textbf{$\Gamma$} & \textbf{transform as} & \textbf{\# of $\gamma$'s} & \textbf{\# of composants}\\\hline
1 & scalar & 0 & 1 \\\hline
$\gamma^\mu$ & vector & 1 & 4\\\hline
$\sigma \gamma^\nu$ & tensor & 2 & 6\\\hline
$\gamma^5 \gamma^\mu$ & axial vector & 3 & 4\\\hline
$\gamma^5$ & pseudoscalar & 4 & 1\\\hline
\end{tabular}
\end{center}
	\item 
This exhausts all possibilities. The total number of components is 16, meaning that the set $\{ 1-\gamma^\mu-\sigma \gamma^\nu-\gamma^5 \gamma^\mu-\gamma^5\}$
	\item 
Makes a complete basis for any four-by-four
matrix.
	\item 
Such $\overline{\Psi} \Gamma\Psi$ currents are called \textbf{bilinear covariants}.
\end{itemize}
\end{frame}

% \subsection{Gammma representation}
%%\section{Definition}
% --------------------------------------------------- Slide --
%\subsection{Definition}
\label{definition}
\begin{frame}\frametitle{Dirac representation}
$\gamma ^{0}={\begin{pmatrix}I_{2}&0\\0&-I_{2}\end{pmatrix}},\quad \gamma ^{k}={\begin{pmatrix}0&\sigma ^{k}\\-\sigma ^{k}&0\end{pmatrix}},\quad \gamma ^{5}={\begin{pmatrix}0&I_{2}\\I_{2}&0\end{pmatrix}}$
Dirac representation
 
The gamma matrices we have written so far are appropriate for acting on Dirac spinors written in the Dirac basis; in fact, the Dirac basis is defined by these matrices. To summarize, in the Dirac basis:


 
\end{frame}
\begin{frame}\frametitle{Majorana representation}
${\displaystyle {\begin{aligned}\gamma ^{0}&={\begin{pmatrix}0&\sigma ^{2}\\\sigma ^{2}&0\end{pmatrix}},&\gamma ^{1}&={\begin{pmatrix}i\sigma ^{3}&0\\0&i\sigma ^{3}\end{pmatrix}},&\gamma ^{2}&={\begin{pmatrix}0&-\sigma ^{2}\\\sigma ^{2}&0\end{pmatrix}},\\\gamma ^{3}&={\begin{pmatrix}-i\sigma ^{1}&0\\0&-i\sigma ^{1}\end{pmatrix}},&\gamma ^{5}&={\begin{pmatrix}\sigma ^{2}&0\\0&-\sigma ^{2}\end{pmatrix}},&C&={\begin{pmatrix}0&-i\sigma ^{2}\\-i\sigma ^{2}&0\end{pmatrix}},\end{aligned}}}$
\\

 (The reason for making all gamma matrices imaginary is solely to obtain the particle physics metric $(+, −, −, −)$, in which squared masses are positive. The Majorana representation, however, is real. One can factor out the $i$ to obtain a different representation with four component real spinors and real gamma matrices. The consequence of removing the $i$ is that the only possible metric with real gamma matrices is $(−, +, +, +)$.)

\end{frame}
\begin{frame}\frametitle{ Weyl (chiral) representation }
Another common choice is the Weyl or chiral basis,[5] in which ${\displaystyle \gamma ^{k}}\gamma ^{k}$ remains the same but ${\displaystyle \gamma ^{0}}\gamma ^{0}$ is different, and so ${\displaystyle \gamma ^{5}}\gamma ^{5}$ is also different, and diagonal,


$\gamma ^{0}={\begin{pmatrix}0&I_{2}\\I_{2}&0\end{pmatrix}},\quad \gamma ^{k}={\begin{pmatrix}0&\sigma ^{k}\\-\sigma ^{k}&0\end{pmatrix}},\quad \gamma ^{5}={\begin{pmatrix}-I_{2}&0\\0&I_{2}\end{pmatrix}},$


${\displaystyle \psi _{L}={\frac {1}{2}}\left(1-\gamma ^{5}\right)\psi ={\begin{pmatrix}I_{2}&0\\0&0\end{pmatrix}}\psi ,\quad \psi _{R}={\frac {1}{2}}\left(1+\gamma ^{5}\right)\psi ={\begin{pmatrix}0&0\\0&I_{2}\end{pmatrix}}\psi .}$

The Weyl basis has the advantage that its chiral projections take a simple form,




\end{frame}

\section{Conclusion}

%========================
% bibliography
%========================
%%%%%%%%%%%%%%%%%%
%
% bibliography
%
%%%%%%%%%%%%%%%%%%

\begin{frame} \frametitle{References}
\begin{thebibliography}{xx}\footnotesize

\bibitem{Rafal Ablamowicz,Pertti Lounesto} {\sc Rafal Ablamowicz  \& Pertti Lounesto}, 1919-1992, {\em Clifford Algebras and
Spinor Structures }, Springer-Science+Business Media, B.V, pp.\ 2--9.

\bibitem{fermions} {\sc V.B.BERESTETSKII, E.M.LIFSHITZL \&.P.PITAEVSKII}, 1982, {\em Quantum Electrodynamics}, Butterworth-Heinemann, pp.~75--81.

\bibitem{VRB}{{\sc Van Rooij JMM \& Bodlaender HL}, 2011, {\em Exact algorithms for dominating set}, Discrete Applied Mathematics, {\bf 159}, pp.\ 2147--2164.}

\end{thebibliography}
\end{frame}



%========================
% thanks
%========================
\begin{frame}\frametitle{Finaly...}
\centering

Thank you for your attention!\\
Any Questions?
\end{frame}



%=================================================
% end presentation
%=================================================
\end{document}
