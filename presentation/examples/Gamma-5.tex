%\section{Description}
% --------------------------------------------------- Slide --
%\subsection{Description}
\label{The fifth Gamma matrix}
\begin{frame}\frametitle{The fifth Gamma matrix}

Define an additional γ-matrix by:\\

\begin{center}
 $\gamma ^{5}={\frac {i}{4!}} . \varepsilon _{\mu \nu \alpha \beta }.\gamma ^{\mu }\gamma ^{\nu }\gamma ^{\alpha }\gamma ^{\beta }$
\newline\newline
$ \gamma ^{5}=i \gamma^{0} \gamma^{1} \gamma^{2} \gamma^{3} ={\begin{pmatrix}0&0&1&0\\0&0&0&1\\1&0&0&0\\0&1&0&0\end{pmatrix}}$

 
\end{center}


\begin{itemize}
 \item  $(\gamma^5)^2 = 1$
\item anti-commutes with every other $\gamma$:
\begin{center}
	$\{\gamma^5,\gamma^\mu\}=0 \Longrightarrow \gamma^5\times \gamma^\mu = - \gamma^\mu \times \gamma^5 $
\end{center}
 



\item
$\operatorname {tr} \left(\gamma ^{5}\right)=\operatorname {tr} \left(\gamma ^{\mu }\gamma ^{\nu }\gamma ^{5}\right)=0$
\end{itemize}

\end{frame}



\label{The fifth Gamma matrix}

\begin{frame}\frametitle{The fifth Gamma matrix}

\begin{center}
	
\begin{align}

$\overline{\Psi} \gamma^5 \Psi  \, \, \Longrightarrow (P\Psi)^{+} \gamma^0 \gamma^5  (P\Psi) \\  \, \, \, \, \,  \Longrightarrow \Psi^{+} (\gamma^0 )^{+} \gamma^5 \gamma^0 \Psi \\  \, \, \, \, \,  \Longrightarrow - \Psi^{+}  (\gamma^0 )^{+}  \gamma^5  \Psi \\ \Longrightarrow  - \overline{\Psi} \gamma^5 \Psi \\  \, \, \, \, \, \Longrightarrow $ Pseudo-scalar 
  
\end{align}
\end{center}

\\\\

{\color{red} To describe the parity-violating in weak interaction, we could (and will!) mix vector and axial interactions.}

\begin{center}
$(\overline{\Psi}  \gamma^\mu \Psi) \pm (\overline{\Psi}  \gamma^\mu \gamma^5 \Psi)$\end{center}




\end{frame}
